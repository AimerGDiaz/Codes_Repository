\documentclass{article}
\usepackage[utf8]{inputenc}
\usepackage{listings}
\usepackage[explicit]{titlesec}
\usepackage[spanish, english]{babel}
\usepackage[hyphens]{url}
 
 \titleformat{\section}
 {\normalfont\normalsize\bfseries\centering}{}{0em}{\textsc{#1}}
 
 \title{\textbf{Comandos utiles de vim}\vspace{-5ex}}
  \date{\vspace{-1ex}}

\begin{document}
\maketitle
\noindent
En overleaf pueden editar sus proyectos utilizando vim, igualmente algunos comandos pueden ejecutarse desde este medio, como los comandos de sustitución, para ello deben ir a settings y seleccionar a vim en el menu de Editor Mode (Key Map)
\begin{enumerate}
\item Delete 
\begin{lstlisting}[language=bash]
d # 
dd 
\end{lstlisting}
\item Move to a particular line
 \begin{lstlisting}[language=bash]
 : # 
 \end{lstlisting}
\item Undo , Redo  and repeat 
\begin{lstlisting}[language=bash]
u , U (undo all the changes),  ctrl +R , . 
\end{lstlisting}
\item Introduce several characters simultaneously 
\begin{lstlisting}[language=bash]
 Ctrl + V , Select lines,  shift + i, \t or # 
\end{lstlisting}
\item  Introduce commands from the vim panel to outside the script  
\begin{lstlisting}[language=bash]
! cat <some file> 
! ls -lh 
 \end{lstlisting}

execute codes but for the current open file  
\begin{lstlisting}[language=bash]
! perl %
! wc %
 \end{lstlisting}
\item  Pattern matching 
\begin{lstlisting}[language=bash]
%s/<pattern syntax like perl>/<replace by>/g(lobal)  or c(case by case) 
n follow
N previous 
\end{lstlisting}
\item  Save and quite, save
\begin{lstlisting}[language=bash]
w! (save independent of the vim mode - like read only mode) 
w save only in read and write mode
wq wirte and quit 
q!  quite without save 
\end{lstlisting}
\item  Insert x times a pattern 

In normal mode (no editing, no command) 
\begin{lstlisting}[language=bash]
x times <character or word> then insert and then escape 
50GACG

\end{lstlisting}
\item  Two or more  screens
\begin{lstlisting}[language=bash]
:sp (horizontal split) 
:sp ./ split the current directory as a file of directories
:vsp  (vertical split)

ctrl + w + down move forward  + up move backward
ctrl + w _ reduce the size of the current window
ctrl + shift + = increase
ctrl + w + r MOVE TO THE TOP THE CURRENT WINDOW
\end{lstlisting}

\item resize 
\begin{lstlisting}
:res number chance the height to "number" rows
:res +number increment "number" rows
:res -number  decrease "number" rows
:vertical resize The same for columns

\end{lstlisting}

\item word movement
\begin{lstlisting}[language=bash]
shift + 6 begining of the line
shift + 4 end of the line
b to go to beginning of current or previous word
w to go the beginning of next word
e to go to the end of current or next word
ge to go the end of the previous word
\end{lstlisting}
\item  paste without adding \# automatically
\begin{lstlisting}[language=bash]
:set  paste
\end{lstlisting}
\item  command history
\begin{lstlisting}[language=bash]
q:
\end{lstlisting}
\item  put line numbers
\begin{lstlisting}[language=bash]
:set number
:set nonumber
\end{lstlisting}
\item  Recording mode
\begin{lstlisting}[language=bash]
1. write 1
2. Activate recording mode  qa
3. esc and press yy, then p 3.
4. move to the second 1 and press ctrl + a
5. close recording mode with q
6. press 99 or any number (shit +2 ) and then a
\end{lstlisting}
Automatizing writing for mysql scripts \url{http://www.thegeekstuff.com/2009/01/vi-and-vim-macro-tutorial-how-to-record-and-play/}

\item  copy and paste
\begin{lstlisting}[language=bash]
single line y and then p
select several lines v, y and then p
\end{lstlisting}
\item  word  completion
\begin{lstlisting}[language=bash]
in insert mode + ctrl + n
\end{lstlisting}
\item  show function documentation
\begin{lstlisting}[language=bash]
in edit mode and mayus press k over the function
\end{lstlisting}
\item  Use the mouse for movement
\begin{lstlisting}[language=bash]
:set mouse=a
:set mouse=
\end{lstlisting}
\item Know current file
\begin{lstlisting}[language=bash]
:f
In normal mode: 
1 then Ctrl+G to get the full file path. 
2 then Ctrl+G to get the full file path and the buffer number you currently have open.
\end{lstlisting}

\end{enumerate}

\end{document}
